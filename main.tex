\documentclass[12pt,a4paper]{article}

% --------- Pacotes ---------
\usepackage[T1]{fontenc}
\usepackage[utf8]{inputenc}
\usepackage[brazil]{babel}
\usepackage{geometry}
\usepackage{setspace}

\geometry{margin=2.5cm}
\onehalfspacing

% --------- Macros (dados da capa) ---------
\newcommand{\instituicao}{CENTRO ESTADUAL DE EDUCAÇÃO TECNOLÓGICA PAULA SOUZA\\
Faculdade de Tecnologia Baixada Santista Rubens Lara}

\newcommand{\curso}{Curso Superior de Tecnologia em Ciência de Dados}
\newcommand{\autor}{Luís Felipe Ruas do Nascimento}
\newcommand{\disciplina}{Álgebra Linear -- Algoritmo de Similaridade de Cosseno}
\newcommand{\cidade}{Santos}
\newcommand{\ano}{2025}

\begin{document}

% --------- Capa ---------
\begin{titlepage}
\centering
{\Large \textbf{\instituicao} \par}
\vspace{2.5cm}
{\large \curso \par}
\vspace{3.5cm}
{\LARGE \bfseries \disciplina \par}
\vspace{3.5cm}
{\Large \autor \par}
\vfill
{\Large \cidade~\ano \par}
\end{titlepage}

% ===================== INÍCIO DO DOCUMENTO =====================


\begin{center}
    {\Large \textbf{Relatório Técnico – Aplicação do TF-IDF e Similaridade do Cosseno}}\\[0.5cm]
    
    \textit{Resumo Técnico – Projeto de Ciência de Dados}\\[0.5cm]
\end{center}

% ===================== RESUMO =====================
\section{Resumo:} 

Este relatório apresenta a aplicação do algoritmo \textit{Term Frequency–Inverse Document Frequency} (TF-IDF) em um conjunto de dados relacionado ao futebol, com o objetivo de calcular similaridade entre descrições textuais de jogadores. O método foi implementado em Python com suporte das bibliotecas \texttt{NLTK} e \texttt{Scikit-learn}, realizando pré-processamento, vetorização e cálculo de similaridade pelo cosseno.

% ===================== 1. INTRODUÇÃO =====================
\section{Introdução}
A análise de similaridade textual é amplamente utilizada em sistemas de recomendação, mineração de texto e recuperação de informação. O algoritmo TF-IDF é um dos métodos mais eficientes para converter documentos em vetores numéricos, permitindo a aplicação de métricas matemáticas como a similaridade do cosseno.

Neste estudo, utiliza-se um dataset contendo informações e descrições de jogadores de futebol, com a finalidade de identificar semelhanças entre perfis textuais.

% ===================== 2. DESCRIÇÃO DO DATASET =====================
\section{Descrição do Dataset}
O conjunto de dados utilizado possui uma coluna principal com descrições textuais de jogadores, contendo atributos como posição, características técnicas e estilo de jogo. O arquivo foi importado em formato CSV e manipulado por meio da biblioteca \texttt{pandas}.

% ===================== 3. METODOLOGIA =====================
\section{Metodologia}

\subsection{Pré-processamento dos Dados}
O texto foi padronizado por meio das seguintes etapas:
\begin{itemize}
    \item Conversão para letras minúsculas;
    \item Remoção de pontuação utilizando expressões regulares;
    \item Tokenização;
    \item Remoção de \textit{stopwords} em português (\texttt{nltk.corpus.stopwords});
    \item Reconstrução do texto limpo em formato string.
\end{itemize}

\subsection{Aplicação do TF-IDF}
Após o pré-processamento, utilizou-se a classe \texttt{TfidfVectorizer} para transformar cada descrição em um vetor numérico baseado na relevância das palavras no corpus.

\subsection{Similaridade do Cosseno}
A similaridade entre os vetores TF-IDF foi calculada através da métrica de cosseno, gerando uma matriz de similaridade que indica o grau de proximidade semântica entre jogadores.

\[
\mathrm{cos\,sim}(u,v) = \frac{u \cdot v}{\|u\|\,\|v\|}
\quad \textnormal{onde} \quad
u \cdot v = \sum_{i=1}^{n} u_i v_i
\]

% ===================== 4. RESULTADOS =====================
\section{Resultados}
A matriz de similaridade permitiu identificar jogadores com perfis próximos a partir de suas descrições. O sistema demonstrou que textos contendo termos técnicos comuns, como "meia ofensivo" ou "velocidade e drible", apresentaram alta similaridade numérica.

Os valores gerados variaram entre 0 e 1, onde 1 indica total equivalência textual e 0 ausência completa de similaridade. Em relação a angulação, quanto mais perto de 0° mais similaridade terá e quanto mais perto de 90° menos similaridade terá (ortogonal)

% ===================== 5. CONCLUSÃO =====================
\section{Conclusão}
O uso do TF-IDF combinado com a similaridade do cosseno mostrou-se eficiente para análise automática de proximidade descritiva entre jogadores de futebol. O método pode ser expandido para aplicações como recomendação de atletas, agrupamento de perfis e análises comparativas em scouting esportivo.


\end{document}

